\documentclass{article}
\usepackage{enumitem}

\begin{document}

\section*{Accuracy Metrics}

\begin{itemize}[label=--,leftmargin=*]
    \item \textbf{True Positive Rate (TPR):} Measures the proportion of actual legitimate attempts that are correctly identified. High TPR is crucial for ensuring legitimate users can access their devices without hindrance.
    \item \textbf{True Negative Rate (TNR) / Specificity:} Reflects the proportion of actual unauthorized attempts that are correctly identified. A high TNR indicates the system's effectiveness in keeping intruders out.
    \item \textbf{False Positive Rate (FPR):} The rate at which legitimate users are incorrectly identified as unauthorized. Lower FPRs are desirable to minimize inconvenience to legitimate users.
    \item \textbf{False Negative Rate (FNR):} The rate at which unauthorized attempts are mistakenly allowed. Minimizing FNR is critical for security.
    \item \textbf{Precision:} The proportion of positive identifications that were actually correct. Precision complements recall by showing the accuracy of positive predictions.
\end{itemize}

\section*{Security Metrics}

\begin{itemize}[label=--,leftmargin=*]
    \item \textbf{Resistance to Attack Vectors:} Evaluate the system's robustness against common attack vectors like shoulder surfing, smudge attacks, and brute force attempts. This can be assessed through simulated attacks and penetration testing.
    \item \textbf{Entropy:} Measures the unpredictability or randomness of the jumbled number pad patterns. Higher entropy indicates a greater level of security as it makes predicting the keypad layout more difficult for attackers.
\end{itemize}

\section*{Usability Metrics}

\begin{itemize}[label=--,leftmargin=*]
    \item \textbf{Authentication Time:} The total time taken for a user to successfully authenticate. Shorter authentication times indicate better usability but must be balanced against security needs.
    \item \textbf{Error Rate:} The rate at which users make mistakes during authentication (e.g., pressing the wrong key). A lower error rate suggests a more user-friendly interface.
\end{itemize}

\section*{Performance Metrics}

\begin{itemize}[label=--,leftmargin=*]
    \item \textbf{Model Training Time:} The time required to train the machine learning model, which impacts the feasibility of updates and retraining.
    \item \textbf{Inference Time:} The time taken for the model to make an authentication decision during actual use. Faster inference times lead to a smoother user experience.
    \item \textbf{Resource Utilization:} Measures the computational resources (CPU, memory) consumed by the system, which is especially important for mobile devices with limited resources.
    \item \textbf{Scalability:} The system's ability to handle a growing number of users without a significant drop in performance.
\end{itemize}

\section*{Adaptability Metrics}

\begin{itemize}[label=--,leftmargin=*]
    \item \textbf{Re-Training Frequency:} The frequency with which the model needs retraining to adapt to changes in user behavior or to incorporate new data. Less frequent retraining with sustained accuracy is ideal.
    \item \textbf{Sensitivity to User Behavior Variability:} Assesses how well the system adapts to natural variations in user behavior over time, maintaining high accuracy without frequent manual recalibrations.
\end{itemize}

\end{document}
