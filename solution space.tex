\documentclass{article}
\usepackage{enumitem}

\begin{document}

\section*{Solution Space}

\subsection*{Advanced Cryptographic Techniques}
\begin{itemize}[label=--,leftmargin=*]
    \item Implement TLS (Transport Layer Security) 1.3 for secure data transmission, ensuring that all communications between the client device and authentication servers are encrypted and tamper-proof. (if data is stored in the server)
    \item Use AES (Advanced Encryption Standard) 256-bit encryption for data at rest, including stored behavioral data and the ML model, ensuring that even if data is accessed, it remains unreadable without the decryption key.
\end{itemize}

\subsection*{Multi-Factor Authentication (MFA) Integration}
\begin{itemize}[label=--,leftmargin=*]
    \item Leverage TOTP (Time-based One-Time Password) algorithms for generating temporary codes sent to a user's mobile device, adding an extra layer of security beyond the jumbled keypad.
\end{itemize}

\subsection*{Machine Learning Model Security}
\begin{itemize}[label=--,leftmargin=*]
    \item Employing Differential Privacy during the data collection and training phases to ensure that individual user behaviors cannot be reverse-engineered from the model.
    \item Utilize Federated Learning to train ML models on distributed devices, enhancing privacy by not centralizing sensitive user data.
\end{itemize}

\subsection*{Behavioral Analysis and Continuous Authentication}
\begin{itemize}[label=--,leftmargin=*]
    \item Implement keystroke dynamics analysis using techniques like Hidden Markov Models (HMM) or Recurrent Neural Networks (RNNs) to continuously evaluate the typing rhythm and patterns for ongoing authentication.
    \item Use anomaly detection algorithms, such as Isolation Forests or One-Class SVM, to identify deviations from established user behavior patterns, indicating potential unauthorized access.
\end{itemize}

\subsection*{Adaptive Authentication Mechanisms}
\begin{itemize}[label=--,leftmargin=*]
    \item Develop a Context-Aware System using GPS, IP geolocation, and device recognition to dynamically adjust authentication challenges based on the assessed risk level of the access attempt. (future scope)
    \item Integrate a Risk-Based Authentication Engine that applies machine learning to assess the risk of each session in real-time, based on user behavior, device integrity checks, and contextual factors.
\end{itemize}

\subsection*{User Experience Optimization}
\begin{itemize}[label=--,leftmargin=*]
    \item Apply Human-Computer Interaction (HCI) principles to the design of the jumbled number pad, ensuring that it remains intuitive and accessible despite its dynamic nature.
\end{itemize}

\end{document}
